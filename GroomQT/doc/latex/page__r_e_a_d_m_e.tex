\hypertarget{page__r_e_a_d_m_e_projet}{}\subsection{Projet}\label{page__r_e_a_d_m_e_projet}
\hypertarget{page__r_e_a_d_m_e_presentation}{}\subsubsection{Présentation}\label{page__r_e_a_d_m_e_presentation}
Le portier connecté «​ gr\+O\+Om​ » permettra à l\textquotesingle{}occupant du bureau de communiquer sa disponibilité avec des personnes extérieures (visiteurs). Pour cela, il utilise une application soit en version PC soit mobile.

Tout en s\textquotesingle{}intégrant facilement à l\textquotesingle{}environnement, il résout le manque d\textquotesingle{}interface entre les utilisateurs et les bureaux permettant de travailler plus efficacement.

A partir de l\textquotesingle{}application, l\textquotesingle{}occupant informe le visiteur de son état \+: \char`\"{}\+Libre\char`\"{}, \char`\"{}\+Occupé\char`\"{} ou \char`\"{}\+Absent\char`\"{}. Il aura la possibilité d\textquotesingle{}ajouter un message \char`\"{}libre\char`\"{} qui s\textquotesingle{}affichera alors sur l\textquotesingle{}écran du portier. S\textquotesingle{}il le souhaite, il informera le visiteur que celui-\/ci peut \char`\"{}\+Entrer\char`\"{} (ou \char`\"{}\+Entrez\char`\"{}). L\textquotesingle{}occupant dispose d\textquotesingle{}un mode \char`\"{}\+Sonnette\char`\"{} qu\textquotesingle{}il peut activer sur le portier connecté. Dans ce mode, le visiteur pourra sonner via l\textquotesingle{}écran tactile et prévenir l\textquotesingle{}occupant de sa présence.

Version \+: PC Desktop Qt\hypertarget{page__r_e_a_d_m_e_informations}{}\subsubsection{Informations}\label{page__r_e_a_d_m_e_informations}
\begin{DoxyAuthor}{Auteur}
Etienne Valette \href{mailto:valette.etienne@gmail.com}{\tt valette.\+etienne@gmail.\+com} 
\end{DoxyAuthor}
\begin{DoxyDate}{Date}
2020 
\end{DoxyDate}
\begin{DoxyVersion}{Version}
0.\+2 
\end{DoxyVersion}
\begin{DoxySeeAlso}{Voir également}
\href{https://svn.riouxsvn.com/groom/}{\tt https\+://svn.\+riouxsvn.\+com/groom/} 
\end{DoxySeeAlso}
